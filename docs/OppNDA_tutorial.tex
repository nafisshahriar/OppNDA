%% OppNDA Tutorial: From Setup to Regression Analysis
%% Supplementary Material for Elsevier SIMPAT Submission
%% DHMAI Network Research Group, 2026

\documentclass[a4paper,11pt]{article}

% Essential packages
\usepackage[utf8]{inputenc}
\usepackage[T1]{fontenc}
\usepackage{lmodern}
\usepackage[margin=2.5cm]{geometry}
\usepackage{graphicx}
\usepackage{xcolor}
\usepackage{hyperref}
\usepackage{listings}
\usepackage{booktabs}
\usepackage{enumitem}
\usepackage{fancyhdr}
\usepackage{titlesec}
\usepackage{caption}
\usepackage{subcaption}
\usepackage{minted}
\usepackage{tcolorbox}

% Color definitions
\definecolor{oppndablue}{RGB}{41, 98, 255}
\definecolor{codebg}{RGB}{245, 245, 245}
\definecolor{codeframe}{RGB}{200, 200, 200}
\definecolor{tipbg}{RGB}{232, 245, 233}
\definecolor{notebg}{RGB}{255, 249, 196}

% Hyperref setup
\hypersetup{
    colorlinks=true,
    linkcolor=oppndablue,
    urlcolor=oppndablue,
    citecolor=oppndablue,
    pdfauthor={DHMAI Network Research Group},
    pdftitle={End-to-End Workflow and Usage Tutorial for OppNDA}
}

% Code listing style
\lstset{
    basicstyle=\ttfamily\small,
    backgroundcolor=\color{codebg},
    frame=single,
    rulecolor=\color{codeframe},
    breaklines=true,
    breakatwhitespace=true,
    tabsize=4,
    showstringspaces=false,
    captionpos=b,
    aboveskip=10pt,
    belowskip=10pt
}

% Custom tip box
\newtcolorbox{tipbox}{
    colback=tipbg,
    colframe=green!60!black,
    title=Tip,
    fonttitle=\bfseries
}

% Custom note box
\newtcolorbox{notebox}{
    colback=notebg,
    colframe=orange!80!black,
    title=Note,
    fonttitle=\bfseries
}

% Header and footer
\pagestyle{fancy}
\fancyhf{}
\fancyhead[R]{Supplementary Material A:\\
End-to-End Workflow and Usage Tutorial for OppNDA}
\fancyfoot[C]{\thepage}
\renewcommand{\headrulewidth}{0.4pt}
\renewcommand{\footrulewidth}{0.4pt}

% Section formatting
\titleformat{\section}
    {\normalfont\Large\bfseries\color{oppndablue}}
    {\thesection}{1em}{}
\titleformat{\subsection}
    {\normalfont\large\bfseries}
    {\thesubsection}{1em}{}
%\begin{comment}
% Figure placeholder command
\newcommand{\screenshotplaceholder}[2]{%
    \begin{figure}[htbp]
        \centering
        \fbox{\parbox{0.9\textwidth}{%
            \centering
            \vspace{3cm}
            \textbf{[Screenshot Placeholder]}\\[0.5em]
            \texttt{#1}\\[0.5em]
            \small\textit{#2}
            \vspace{3cm}
        }}
        \caption{#2}
        \label{fig:#1}
    \end{figure}
}
%\end{comment}
%% To replace placeholders with actual screenshots, use:
%% \begin{figure}[htbp]
%%     \centering
%%     \includegraphics[width=0.9\textwidth]{screenshots/filename.png}
%%     \caption{Description}
%%     \label{fig:label}
%% \end{figure}

\begin{document}

% Title page
\begin{titlepage}
    \centering
    \vspace*{2cm}
    
    {\Huge\bfseries\color{oppndablue} Supplementary Material A:\\[0.5em]}
    {\LARGE End-to-End Workflow and Usage Tutorial for OppNDA\\[2em]}
    
    \vspace*{\fill}
    
    
\end{titlepage}
\newpage
% Table of contents
\tableofcontents


%% ============================================================================
\section{Introduction}
\label{sec:introduction}
%% ============================================================================

OppNDA (Opportunistic Network Data Analyzer) is a comprehensive framework that streamlines the workflow for configuring, running, and analyzing simulations in the ONE simulator. This document provides a step-by-step walkthrough of the framework, supplementing the main manuscript titled \textbf{OppNDA: A Modular and Scalable Automation Framework for Streamlining DTN Research with the ONE simulator}, submitted to \emph{Simulation Modelling Practice and Theory}.

This supplementary material is intended for reviewers and researchers interested
in reproducing or extending the OppNDA analysis workflow. It emphasizes practical
usage rather than implementation details, which are provided separately in
Supplementary Material B.

\subsection{Workflow Overview}

The complete OppNDA workflow consists of the following stages:

\begin{enumerate}
    \item \textbf{Setup}: Install dependencies and launch the application
    \item \textbf{Configuration}: Define simulation scenarios using the web interface
    \item \textbf{Simulation}: Execute the ONE simulator with configured parameters
    \item \textbf{Averaging}: Aggregate results across multiple simulation parameters
    \item \textbf{Analysis}: Generate visualizations
    \item \textbf{Regression}: Train machine learning models on simulation data
\end{enumerate}

\subsection{Prerequisites}

Before proceeding, ensure you have the following:

\begin{itemize}
    \item Python 3.9 or higher
    \item ONE simulator (for generating simulation reports)
    \item A modern web browser (Chrome, Firefox, Edge)
    \item Git (for cloning the repository)
    \item Docker (optional, for containerized deployment)
\end{itemize}
\newpage
%% ============================================================================
\section{Installation and Setup}
\label{sec:installation}
%% ============================================================================

OppNDA provides multiple installation methods to accommodate different environments and preferences.
\subsection{Method 1: Automated Setup Scripts}

The simplest installation method uses the provided setup scripts.

\subsubsection{Windows}

\begin{enumerate}
    \item Download or clone the OppNDA repository:
    \begin{lstlisting}[language=bash]
git clone https://github.com/*/*.git
cd OppNDA
    \end{lstlisting}
    
    \item Run the setup script:
    \begin{lstlisting}[language=bash]
scripts\setup.bat
    \end{lstlisting}
    
    \item Launch the application:
    \begin{lstlisting}[language=bash]
scripts\start.bat
    \end{lstlisting}
\end{enumerate}

\begin{figure}[htbp]
     \centering
     \includegraphics[width=0.9\textwidth]{screenshots/1.jpg}
     \caption{Running setup.bat on Windows}
     \label{fig:setup_windows}
\end{figure}

\subsubsection{Linux/macOS}

\begin{enumerate}
    \item Clone and navigate to the repository:
    \begin{lstlisting}[language=bash]
git clone https://github.com/*/*.git
cd OppNDA
    \end{lstlisting}
    
    \item Run the setup script:
    \begin{lstlisting}[language=bash]
bash scripts/setup.sh
    \end{lstlisting}
    
    \item Launch the application:
    \begin{lstlisting}[language=bash]
bash scripts/start.sh
    \end{lstlisting}
\end{enumerate}

\subsection{Method 2: Manual Installation}

For more control over the installation process:

\begin{lstlisting}[language=bash]
# Create and activate virtual environment
python -m venv venv
source venv/bin/activate    # Linux/macOS
venv\Scripts\activate       # Windows

# Install dependencies
pip install -r requirements.txt

# Run the application
python OppNDA.py
\end{lstlisting}

\subsection{Method 3: Docker Installation}

For containerized deployment, OppNDA includes Docker support.

\subsubsection{Using Docker Compose (Recommended)}

\begin{lstlisting}[language=bash]
# Build and run with Docker Compose
docker-compose up --build
\end{lstlisting}

\subsubsection{Manual Docker Build}

\begin{lstlisting}[language=bash]
# Build the Docker image
docker build -t oppnda .

# Run the container
docker run -p 5001:5001 --name OppNDA oppnda
\end{lstlisting}

\begin{figure}[htbp]
     \centering
     \includegraphics[width=0.9\textwidth]{screenshots/2.jpg}
     \caption{OppNDA running in a Docker container}
     \label{fig:docker-running}
\end{figure}

\subsection{Accessing the Web Interface}

After launching OppNDA, open your web browser and navigate to:

\begin{center}
    \url{http://localhost:5001/}
\end{center}

\begin{figure}[htbp]
     \centering
     \includegraphics[width=0.9\textwidth]{screenshots/3.jpg}
     \caption{OppNDA web interface home page}
     \label{fig:homepage}
\end{figure}


%% ============================================================================
\section{Scenario Configuration}
\label{sec:configuration}
%% ============================================================================

The Settings page provides a comprehensive interface for configuring ONE simulator scenarios. This section covers both the creation of new configurations and the import of existing ones.

\subsection{Importing Existing Configuration Files}

OppNDA supports importing existing ONE simulator configuration files (\texttt{.txt} format) directly into the GUI.

\begin{enumerate}
    \item Click the \textbf{Import Config} button in the toolbar
    \item Select your existing ONE configuration file
    \item The GUI will parse and populate all settings automatically
    \item Review and modify imported settings as needed
\end{enumerate}

\begin{figure}[htbp]
     \centering
     \includegraphics[width=0.9\textwidth]{screenshots/4.jpg}
     \caption{Importing an existing ONE configuration file}
     \label{fig:import-config}
\end{figure}



\begin{notebox}
When importing, OppNDA automatically recognizes standard ONE parameters and maps them to the corresponding GUI fields. Unknown parameters are preserved and can be viewed in the advanced settings section.
\end{notebox}

\subsection{Scenario Settings}

The basic scenario settings define the simulation environment:

\begin{enumerate}
    \item \textbf{Scenario Name}: Enter a descriptive name for your simulation
    \item \textbf{Simulation Time}: Set the duration in seconds
    \item \textbf{Update Interval}: Configure the simulation tick rate
    \item \textbf{World Size}: Define the simulation area dimensions (X, Y)
\end{enumerate}



\subsection{Network Interfaces}

Configure the communication interfaces available to nodes:

\begin{enumerate}
    \item Click \textbf{Add Interface} to create a new interface
    \item Select the interface type (e.g., SimpleBroadcastInterface)
    \item Configure parameters:
    \begin{itemize}
        \item Transmission speed
        \item Transmission range
    \end{itemize}
\end{enumerate}


\begin{figure}[htbp]
     \centering
     \includegraphics[width=0.9\textwidth]{screenshots/5.jpg}
     \caption{Network interface configuration}
     \label{fig:interface-config}
\end{figure}


\subsection{Host Groups}

Define different categories of mobile nodes:

\begin{enumerate}
    \item Click \textbf{Add Group} to create a new host group
    \item Configure group parameters:
    \begin{itemize}
        \item Group ID and number of hosts
        \item Movement model (RandomWaypoint, ShortestPathMapBasedMovement, etc.)
        \item Buffer size
        \item Message TTL (Time-To-Live)
        \item Router type (Epidemic, SprayAndWait, PRoPHET, etc.)
    \end{itemize}
\end{enumerate}


\begin{figure}[htbp]
     \centering
     \includegraphics[width=0.9\textwidth]{screenshots/7.jpg}
     \caption{Host group configuration panel}
     \label{fig:group-config}
\end{figure}



\subsection{Message Events}

Configure message generation patterns:

\begin{enumerate}
    \item Click \textbf{Add Event} to create a new event generator
    \item Set event parameters:
    \begin{itemize}
        \item Event class (MessageEventGenerator)
        \item Message creation interval (min, max)
        \item Message size range
        \item Source and destination host ranges
        \item Event timing (start, end)
    \end{itemize}
\end{enumerate}

\begin{figure}[htbp]
     \centering
     \includegraphics[width=0.9\textwidth]{screenshots/msg.jpg}
     \caption{Message event generator configuration}
     \label{fig:event-config}
\end{figure}

\subsection{Report Selection}

Choose which reports the ONE simulator should generate:

\begin{enumerate}
    \item Browse the available report types
    \item Select desired reports (e.g., MessageStatsReport, DeliveredMessagesReport)
    \item Configure report-specific parameters if needed
    \item Set the report output directory
\end{enumerate}


\begin{figure}[htbp]
     \centering
     \includegraphics[width=0.9\textwidth]{screenshots/8.jpg}
     \caption{Report type selection panel}
     \label{fig:report-selection}
\end{figure}


\subsection{Saving Configuration}

After configuring all parameters:

\begin{enumerate}
    \item Click \textbf{Save Config} to save the current configuration
    \item The configuration is saved as a \texttt{.txt} file compatible with ONE
    \item OppNDA also maintains JSON backups of all settings
\end{enumerate}

%% ============================================================================
\section{Running the ONE Simulator}
\label{sec:running}
%% ============================================================================

OppNDA provides integrated execution of the ONE simulator with real-time output monitoring.

\subsection{Starting a Simulation}

\begin{enumerate}
    \item Ensure your configuration is saved
    \item Click the \textbf{Run ONE} button
    \item OppNDA will:
    \begin{itemize}
        \item Save the current configuration
        \item Launch the ONE simulator with appropriate parameters
        \item Display real-time console output
        \item Automatically trigger post-processing upon completion
    \end{itemize}
\end{enumerate}

\subsection{Monitoring Progress}

The console panel displays real-time output from the simulator:

\begin{itemize}
    \item Simulation progress percentage
    \item Current simulation time
    \item Messages created and delivered
    \item Any warnings or errors
\end{itemize}


\begin{figure}[htbp]
     \centering
     \includegraphics[width=0.9\textwidth]{screenshots/9.jpg}
     \caption{Real-time console output during simulation}
     \label{fig:report-selection}
\end{figure}


\begin{tipbox}
For batch simulations with multiple parameter combinations, you can configure seed ranges and parameter variations. OppNDA will execute simulations sequentially and aggregate results.
\end{tipbox}

\subsection{Terminating a Simulation}

If needed, you can stop a running simulation:

\begin{enumerate}
    \item Click the \textbf{Terminate} button next to the Run ONE button
    \item Confirm the termination
    \item Any partial results will be preserved
\end{enumerate}

%% ============================================================================
\section{Post-Processing: Report Averaging}
\label{sec:averaging}
%% ============================================================================

The averaging module aggregates results across multiple simulation runs (seeds) to produce statistically meaningful data.

\subsection{Understanding Filename Patterns}

OppNDA uses configurable filename patterns to group and average reports. An example of report filename follows this structure:

\begin{lstlisting}
RouterType_TTL_BufferSize_Seed_ReportType.txt
\end{lstlisting}

For example: \texttt{Epidemic\_300\_5M\_1\_MessageStatsReport.txt}

\subsection{Configuring the Average class }

\begin{enumerate}
    \item Navigate to the \textbf{Post-Processing} section
    \item Set the \textbf{Input Directory} containing raw report files
    \item Configure the \textbf{Filename Pattern} using the drag-and-drop pattern builder
    \item Set \textbf{Grouping Parameters} to define which parameters to group by (e.g., Router, TTL, Buffer)
    \item Set the \textbf{Seed Position} to identify which part of the filename represents the seed number
\end{enumerate}

\begin{figure}[htbp]
     \centering
     \includegraphics[width=0.9\textwidth]{screenshots/10.jpg}
     \caption{\textit{Averager} configuration panel}
     \label{fig:Averager-config}
\end{figure}


\subsection{Pattern Builder}

The interactive pattern builder helps define filename patterns:

\begin{enumerate}
    \item Drag components (Router, TTL, Buffer, Seed, etc.) into position
    \item Set delimiters between components (underscore, hyphen, etc.)
    \item Preview how patterns match your actual files
\end{enumerate}

\begin{figure}[htbp]
     \centering
     \includegraphics[width=0.9\textwidth]{screenshots/11.jpg}
     \caption{Interactive pattern builder}
     \label{fig:pattern-builder}
\end{figure}


\subsection{Running the 'Averager'}

\begin{enumerate}
    \item Click \textbf{Run \textit{Averager}}
    \item Monitor progress in the console panel
    \item Averaged files are saved with \texttt{averaged\_} prefix in the output directory
\end{enumerate}


\begin{figure}[htbp]
     \centering
     \includegraphics[width=0.9\textwidth]{screenshots/12.jpg}
     \caption{\textit{Averager} console output showing grouped files}
     \label{fig:Averagers-output}
\end{figure}


\subsection{Understanding Averaged Output}

The averaged output files contain:

\begin{itemize}
    \item Mean values across all seeds
    \item Standard deviation for each metric
    \item Count of samples averaged
\end{itemize}

%% ============================================================================
\section{Post-Processing: Visualization and Analysis}
\label{sec:analysis}
%% ============================================================================

The analysis module generates publication-ready visualizations from averaged data.

\subsection{Configuring Analysis}

\begin{enumerate}
    \item Select the \textbf{Input Directory} containing averaged reports
    \item Choose \textbf{Report Types} to analyze
    \item Configure axis parameters (X, Y, Z variables)
    \item Set desired \textbf{Plot Types}
\end{enumerate}


\begin{figure}[htbp]
     \centering
     \includegraphics[width=0.9\textwidth]{screenshots/13.jpg}
     \caption{Analysis configuration panel}
     \label{fig:analysis-config}
\end{figure}


\subsection{Available Plot Types}

OppNDA supports multiple visualization types:

\begin{itemize}
    \item \textbf{3D Surface Plots}: Visualize relationships between three variables
    \item \textbf{Line Plots}: Show trends across parameter values
    \item \textbf{Heatmaps}: Display intensity matrices
    \item \textbf{Violin Plots}: Show data distribution
    \item \textbf{Pair Plots}: Explore correlations between metrics
\end{itemize}


\subsection{Plot Settings}

Customize plot appearance:

\begin{enumerate}
    \item \textbf{Figure Size}: Set dimensions in inches
    \item \textbf{Font Sizes}: Configure title, label, and tick fonts
    \item \textbf{Color Scheme}: Choose from available color palettes
    \item \textbf{DPI}: Set resolution for saved images
\end{enumerate}

\subsection{Running Analysis}

\begin{enumerate}
    \item Click \textbf{Run Analysis}
    \item Monitor progress as plots are generated
    \item Generated plots are saved to the \texttt{plots/} directory
\end{enumerate}


\begin{figure}[htbp]
     \centering
     \includegraphics[width=0.9\textwidth]{screenshots/14.jpg}
     \caption{Analysis progress with real-time logging}
     \label{fig:analysis-running}
\end{figure}


\subsection{Viewing Generated Plots}

After analysis completes:

\begin{itemize}
    \item View plots directly in the web interface
    \item Access high-resolution versions in the \texttt{plots/} directory
    \item Download plots in PNG or PDF format
\end{itemize}

\begin{figure}[htbp]
     \centering
     \includegraphics[width=0.9\textwidth]{screenshots/r1.jpg}
     \caption{View results generated from simulation reports}
     \label{fig:res}
\end{figure}

\begin{figure}[htbp]
     \centering
     \includegraphics[width=0.9\textwidth]{screenshots/r2.jpg}
     \caption{Visualizations of simulation data}
     \label{fig:res1}
\end{figure}

%% ============================================================================
\section{Regression Analysis}
\label{sec:regression}
%% ============================================================================

The regression module trains machine learning models to understand and predict network performance metrics.

\subsection{Preparing Input Data}

Regression analysis uses CSV files generated during the analysis phase:

\begin{enumerate}
    \item Ensure analysis has been run to generate CSV output
    \item The CSV files contain structured data suitable for ML training
\end{enumerate}

\subsection{Configuring Regression}

\begin{enumerate}
    \item Navigate to the \textbf{Regression} section
    \item Select the \textbf{Input CSV File}
    \item Choose \textbf{Target Variable(s)} to predict (e.g., delivery\_ratio, latency)
    \item Select \textbf{Predictor Variables} (e.g., TTL, buffer\_size, num\_hosts)
    \item Choose \textbf{ML Models} to train
\end{enumerate}


\begin{figure}[htbp]
     \centering
     \includegraphics[width=0.9\textwidth]{screenshots/15.jpg}
     \caption{Regression configuration panel}
     \label{fig:regression-config}
\end{figure}

\begin{figure}[htbp]
     \centering
     \includegraphics[width=0.9\textwidth]{screenshots/16.jpg}
     \caption{Advanced regression configuration}
     \label{fig:regression-config}
\end{figure}


\subsection{Running Regression}

\begin{enumerate}
    \item Click \textbf{Run Regression}
    \item Monitor training progress
    \item View model performance metrics ($R^2$, RMSE, MAE)
\end{enumerate}


\begin{figure}[htbp]
     \centering
     \includegraphics[width=0.9\textwidth]{screenshots/17.jpg}
     \caption{Regression training progress}
     \label{fig:regression-running}
\end{figure}

\begin{figure}[htbp]
     \centering
     \includegraphics[width=0.9\textwidth]{screenshots/18.jpg}
     \caption{Feature importance of spray and wait router using random forest}
     \label{fig:regression-running}
\end{figure}

\begin{tipbox}
Use feature importance to identify which simulation parameters most strongly affect network performance metrics. This can inform future parameter selection for simulations.
\end{tipbox}


\end{document}